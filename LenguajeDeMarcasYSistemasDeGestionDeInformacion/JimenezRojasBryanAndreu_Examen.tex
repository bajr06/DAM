\documentclass{article}
\usepackage{graphicx} % Required for inserting images
\usepackage{amsmath}
\usepackage{amssymb}

\title{EXAMEN 2024}
\author{Bryan Andreu Jiménez Rojas}
\date{November 2024}

\begin{document}
\maketitle

\section{Potencia Compleja}
\text{
    En este primer ejemplo, tenemos la expresión de una potencia compleja $x^{y}$. La fórmula general para una potencia compleja es:
}

\begin{equation*}
    z^{n}=r^{n} \left(\cos{\left(n\theta\right)} + i \sin{\left(n\theta\right)}\right)
\end{equation*}

\text{donde $z=r\left(\cos{\left(\theta\right)} + i \sin{\left(\theta\right)}\right)$ es la forma de un número complejo.}


\section{Fracción con Varias Partes}
\begin{equation}
    \frac{1+e^{i\theta}}{1-e^{-i\theta}} = \frac{\cos{\left(\theta\right)} + i \sin{\left(\theta\right)} + 1}{\cos{\left(\theta\right)} - i \sin{\left(\theta\right)} - 1}
\end{equation}


\section{Límite}
\begin{equation}
    \lim_{z\xrightarrow{}0}\frac{e^{iz} - 1}{z} = i
\end{equation}

\text{Este es un límite clásico que describe la derivada de $e^{iz}$ en $z = 0$.}


\section{Sumatorio}
\begin{equation*}
    \sum_{n=1}^{\infty}\frac{1}{n^{2}} = \frac{\pi^{2}}{6}
\end{equation*}

\text{
Este es un resultado conocido como la serie de Basilea, que es importante en la teoría de series infinitas.
}

\end{document}

\documentclass{beamer}

\usepackage[spanish]{babel}
\usepackage[utf8]{inputenc}
\usepackage[T1]{fontenc}
\usefonttheme{professionalfonts}
\usetheme{Madrid}
\usecolortheme{default}

\title{Virtualización con VMware y Ubuntu}
\subtitle{Parte 3/4}
\author{Víctor de Juan}
\date{\today}

\begin{document}

%----------------------------------
% Portada
%----------------------------------
\begin{frame}
	\titlepage
\end{frame}

%----------------------------------
% Índice de la Parte 3
%----------------------------------
\begin{frame}{Índice - Parte 3/4}
	\tableofcontents[hideallsubsections]
\end{frame}

%=====================================================================
% 11. Instalación de Ubuntu en la Máquina Virtual
%=====================================================================
\section{Instalación de Ubuntu en la Máquina Virtual}

%----------------------------------
\subsection{11.1 Pantalla de Bienvenida: Elegir Idioma y Modo de Instalación}
\begin{frame}{Pantalla de Bienvenida (I)}
	\begin{itemize}
		\item Al iniciar la VM con la ISO de Ubuntu, aparece un menú para:
			\begin{itemize}
				\item \textbf{Probar Ubuntu} (modo live) sin instalar.
				\item \textbf{Instalar Ubuntu} directamente en el disco virtual.
			\end{itemize}
		\item Selecciona \textbf{Instalar Ubuntu} para comenzar el proceso de instalación.
	\end{itemize}
\end{frame}

\begin{frame}{Pantalla de Bienvenida (II)}
	\begin{itemize}
		\item \textbf{Idioma}: Elige el idioma para el instalador y la futura configuración del sistema.
		\item \textbf{Teclado}: Se puede cambiar aquí o más adelante. Verifica si usas teclados en español (España o Latinoamérica).
		\item \textbf{Opciones de accesibilidad}: Voz, alto contraste, teclas lentas, etc., disponibles desde el inicio.
	\end{itemize}
\end{frame}

%----------------------------------
\subsection{11.2 Configuración del Teclado y la Red}
\begin{frame}{Configuración del Teclado}
	\begin{itemize}
		\item Antes de empezar la instalación, Ubuntu pide confirmar la \textbf{distribución de teclado}.
		\item Puedes probar las teclas en el cuadro de texto para evitar errores posteriores.
		\item Ejemplo: \textit{Español (España)}, \textit{Español (Latinoamérica)}, \textit{Inglés (US)}, etc.
	\end{itemize}
\end{frame}

\begin{frame}{Configuración de la Red}
	\begin{itemize}
		\item En algunas versiones, se pide configurar la red \textbf{(Wi-Fi, Ethernet o proxy)}.
		\item En una VM, la conexión suele estar vía NAT/Bridged, así que el instalador por lo general ya tiene acceso a la red.
		\item Si no hay DHCP, se puede configurar IP estática manualmente (en versiones server).
	\end{itemize}
\end{frame}

%----------------------------------
\subsection{11.3 Instalación Normal vs Instalación Mínima}
\begin{frame}{Instalación Normal}
	\begin{itemize}
		\item Incluye escritorio completo (GNOME o el flavor elegido), navegador, aplicaciones básicas (LibreOffice, etc.).
		\item \textbf{Ideal} para usuarios de escritorio o laboratorios de ofimática.
		\item Tamaño mayor en disco y más paquetes instalados.
	\end{itemize}
\end{frame}

\begin{frame}{Instalación Mínima}
	\begin{itemize}
		\item Instala \textbf{menos software} por defecto: Navegador y utilidades esenciales.
		\item \textbf{Recomendado} si se desea un entorno más ligero o si piensas instalar solo los paquetes necesarios.
		\item Ahorra espacio en disco y reduce el tiempo de instalación.
	\end{itemize}
\end{frame}

%----------------------------------
\subsection{11.4 Descargar Actualizaciones Durante la Instalación}
\begin{frame}{Descargar Actualizaciones Durante la Instalación}
	\begin{itemize}
		\item Puedes marcar la casilla para \textbf{descargar e instalar actualizaciones} en el proceso de instalación.
		\item \textbf{Ventaja}: El sistema estará actualizado inmediatamente tras el primer arranque.
		\item \textbf{Desventaja}: Requiere buena conexión a internet y aumenta el tiempo de instalación.
	\end{itemize}
\end{frame}

%----------------------------------
\subsection{11.5 Tipos de Instalación (Automática vs Manual)}
\begin{frame}{Instalación Automática}
	\begin{itemize}
		\item El instalador gestiona el particionado y el \textbf{layout de disco}.
		\item Opción típica: “Borrar disco e instalar Ubuntu” (en la VM, el “disco” es virtual).
		\item Crea particiones automáticamente (a menudo / y swap, o / y un archivo swap en versiones recientes).
	\end{itemize}
\end{frame}

\begin{frame}{Instalación Manual}
	\begin{itemize}
		\item Permite \textbf{personalizar particiones}, sistemas de archivos y puntos de montaje.
		\item Recomendado en entornos de producción, donde quizás quieras particionar /home, /boot/efi, etc.
		\item Mayor control, pero exige conocimiento de particiones y del proceso de arranque.
	\end{itemize}
\end{frame}

%=====================================================================
% 12. Particionamiento en Profundidad
%=====================================================================
\section{Particionamiento en Profundidad}

%----------------------------------
\subsection{12.1 Particiones Recomendadas en Linux}
\begin{frame}{Particiones Recomendadas (I)}
	\begin{itemize}
		\item \textbf{/ (Root)}: Donde se instalan los archivos principales del SO.
			\begin{itemize}
				\item Tamaño sugerido: 15-20 GB mínimo para un entorno desktop, más si se instalan muchos programas.
			\end{itemize}
		\item \textbf{/home}: Directorio de los usuarios, opcional separarlo para resguardar datos en reinstalaciones.
	\end{itemize}
\end{frame}

\begin{frame}{Particiones Recomendadas (II)}
	\begin{itemize}
		\item \textbf{swap}: Espacio en disco usado como “memoria virtual” cuando se agota la RAM.
			\begin{itemize}
				\item Regla general: ~ la misma cantidad que la RAM si se pretende hibernar; de lo contrario, con 2-4 GB puede bastar.
			\end{itemize}
		\item \textbf{/boot/efi (UEFI)}: En sistemas UEFI, aquí se almacenan los archivos de arranque. Suele ser FAT32 con ~100-300 MB.
	\end{itemize}
\end{frame}

%----------------------------------
\subsection{12.2 Particionamiento Automático (Ubiquity)}
\begin{frame}{12.2 Particionamiento Automático}
	\begin{itemize}
		\item \textbf{Ubiquity} (instalador estándar de Ubuntu Desktop) ofrece una opción de “Borrar disco e instalar Ubuntu” que crea las particiones por defecto.
		\item En Ubuntu Server, el instalador también ofrece un modo automático de particionado.
		\item \textbf{Ventaja}: Sencillez y rapidez.
		\item \textbf{Desventaja}: Menos flexibilidad en la distribución del espacio.
	\end{itemize}
\end{frame}

%----------------------------------
\subsection{12.3 Particionamiento Manual}
\begin{frame}{Particionamiento Manual (I)}
	\begin{itemize}
		\item El usuario define cada partición: tamaño, tipo, sistema de archivos, punto de montaje.
		\item \textbf{MBR} vs \textbf{GPT} (depende del firmware seleccionado en la VM).
		\item Creación de particiones primarias, extendidas y lógicas (si usas MBR).
	\end{itemize}
\end{frame}

\begin{frame}{Particionamiento Manual (II)}
	\textbf{Selección de sistemas de archivos:}
	\begin{itemize}
		\item \textbf{ext4}: Predeterminado en Ubuntu, estable y maduro.
		\item \textbf{xfs}: Uso frecuente en servidores, excelente en grandes volúmenes de datos.
		\item \textbf{btrfs}: Funciones avanzadas (snapshot, subvolúmenes), pero menos usado por defecto en Ubuntu.
	\end{itemize}
	\vspace{0.2cm}
	\textbf{Opciones avanzadas}:
	\begin{itemize}
		\item \textbf{LVM} (Logical Volume Manager): Facilita redimensionar o crear volúmenes lógicos en caliente.
	\end{itemize}
\end{frame}

%----------------------------------
\subsection{12.4 Instalación del Gestor de Arranque (GRUB)}
\begin{frame}{GRUB en BIOS y UEFI (I)}
	\begin{itemize}
		\item \textbf{GRUB}: Programa que gestiona el inicio del sistema operativo.
		\item En \textbf{modo BIOS} (MBR), GRUB se instala en el MBR (primer sector del disco).
		\item En \textbf{modo UEFI}, GRUB se instala en la EFI System Partition (ESP) y se registra en la NVRAM.
	\end{itemize}
\end{frame}

\begin{frame}{GRUB en BIOS y UEFI (II)}
	\begin{itemize}
		\item Durante la instalación, Ubuntu detecta si estás en modo \textbf{Legacy BIOS} o \textbf{UEFI} y ubica GRUB según corresponda.
		\item Si hay más de un SO, GRUB añade un menú para elegir el sistema a arrancar (dual-boot).
		\item \textbf{Verificación:} Tras reiniciar, el sistema debe arrancar Ubuntu directamente. Si no, revisar la configuración de firmware o el orden de arranque.
	\end{itemize}
\end{frame}

%=====================================================================
% 13. Primer Inicio de Ubuntu
%=====================================================================
\section{Primer Inicio de Ubuntu}

%----------------------------------
\subsection{13.1 Configuración Inicial (Usuarios, Idioma, Zona Horaria)}
\begin{frame}{Configuración Inicial (I)}
	\begin{itemize}
		\item Tras la instalación, al primer arranque se crea/valida el \textbf{usuario principal}.
		\item Escoge \textbf{contraseña} y nombre de la cuenta (cuenta con permisos de sudo en entornos desktop).
		\item \textbf{Zona horaria}: Se configura automáticamente por geolocalización (si está habilitada) o se selecciona manualmente.
	\end{itemize}
\end{frame}

\begin{frame}{Configuración Inicial (II)}
	\begin{itemize}
		\item \textbf{Opciones de privacidad}: Ubuntu puede consultar si deseas enviar estadísticas de uso (opcional).
		\item \textbf{Configuración regional}: Idioma del sistema, formatos de fecha y número, teclado.
		\item \textbf{Aplicaciones iniciales}: Se muestran sugerencias de software o se puede saltar este paso.
	\end{itemize}
\end{frame}

%----------------------------------
\subsection{13.2 Actualización de Paquetes (apt update, apt upgrade)}
\begin{frame}{Actualización de Paquetes}
	\begin{itemize}
		\item Abre la \textbf{terminal} (Ctrl+Alt+T) y ejecuta:
			\begin{itemize}
				\item \texttt{sudo apt update} (actualiza la lista de paquetes)
				\item \texttt{sudo apt upgrade} (descarga e instala actualizaciones disponibles)
			\end{itemize}
		\item Mantener el sistema al día mejora la seguridad y la estabilidad.
		\item \textbf{Opcional}: Usar \texttt{sudo apt dist-upgrade} para recibir cambios más profundos de dependencias.
	\end{itemize}
\end{frame}

%=====================================================================
% 14. Instalación y Configuración de VMware Tools (open-vm-tools)
%=====================================================================
\section{Instalación y Configuración de VMware Tools}

%----------------------------------
\subsection{14.1 open-vm-tools vs VMware Tools propietario}
\begin{frame}{14.1 Diferencias Principales}
	\begin{itemize}
		\item \textbf{open-vm-tools}: Versión de código abierto integrada en los repositorios de la mayoría de distribuciones Linux.
			\begin{itemize}
				\item Mejor integración en el sistema de paquetes (apt en Ubuntu).
				\item Recibe actualizaciones de la comunidad y de VMware.
			\end{itemize}
		\item \textbf{VMware Tools propietario}: Instalable manualmente desde la ISO que trae VMware.
			\begin{itemize}
				\item A veces incluye algunas funciones extra o soporte específico.
			\end{itemize}
	\end{itemize}
\end{frame}

%----------------------------------
\subsection{14.2 Ventajas (mejoras gráficas, sincronización de hora, drag \& drop)}
\begin{frame}{14.2 Ventajas de las VMware Tools}
	\begin{itemize}
		\item \textbf{Mejoras gráficas}: Resolución dinámica, soporte de mouse integrado (no necesidad de “capturar” cursor).
		\item \textbf{Sincronización horaria}: La VM ajusta su reloj con el host para evitar desfases.
		\item \textbf{Carpetas compartidas}: Posibilidad de compartir una carpeta del host con la VM.
		\item \textbf{Portapapeles compartido y drag \& drop}: Copiar y pegar entre host e invitado.
	\end{itemize}
\end{frame}

%----------------------------------
\subsection{14.3 Instalación paso a paso}
\begin{frame}{14.3 Instalación de open-vm-tools}
	\begin{itemize}
		\item \textbf{Ubuntu Desktop}:
			\begin{itemize}
				\item \texttt{sudo apt update}
				\item \texttt{sudo apt install open-vm-tools open-vm-tools-desktop}
			\end{itemize}
		\item \textbf{Ubuntu Server} (sin entorno gráfico):
			\begin{itemize}
				\item \texttt{sudo apt update}
				\item \texttt{sudo apt install open-vm-tools}
			\end{itemize}
		\item Reiniciar la VM o iniciar el servicio para que surta efecto.
	\end{itemize}
\end{frame}

%----------------------------------
\subsection{14.4 Troubleshooting común (pantalla, drivers, etc.)}
\begin{frame}{14.4 Problemas frecuentes}
	\begin{itemize}
		\item \textbf{Resolución de pantalla} no cambia:
			\begin{itemize}
				\item Verificar instalación de \texttt{open-vm-tools-desktop}, ya que incluye los drivers gráficos.
			\end{itemize}
		\item \textbf{Integración de ratón} no funciona:
			\begin{itemize}
				\item Asegurarse de no tener otro driver de virtualización en conflicto (ej. spice-vdagent, virtualbox-guest-utils, etc.)
			\end{itemize}
		\item \textbf{Sin carpetas compartidas}:
			\begin{itemize}
				\item Comprobar que la función de “Shared Folders” esté activada en la configuración de la VM (solo en Workstation Pro).
			\end{itemize}
	\end{itemize}
\end{frame}

%=====================================================================
% 15. Configuraciones Avanzadas en VMware
%=====================================================================
\section{Configuraciones Avanzadas en VMware}

%----------------------------------
\subsection{15.1 Redes Virtuales Avanzadas}
\begin{frame}{vmnet0, vmnet1, vmnet8 y redes personalizadas}
	\begin{itemize}
		\item \textbf{vmnet0 (Bridged)}: Conecta la VM directamente a la misma red que el host físico.
		\item \textbf{vmnet1 (Host-Only)}: Red interna entre el host y las VMs, sin acceso a internet.
		\item \textbf{vmnet8 (NAT)}: El host actúa como un router NAT, compartiendo su conexión con la VM.
		\item \textbf{Redes personalizadas}: En Workstation Pro, se pueden crear redes extra con rangos específicos, DHCP configurado, etc.
	\end{itemize}
\end{frame}

\begin{frame}{VLAN Trunking y NAT Port Forwarding}
	\begin{itemize}
		\item \textbf{VLAN Trunking}: Posible en versiones Pro, para pasar etiquetas VLAN desde el host a la VM.
		\item \textbf{NAT Port Forwarding}: Reenvío de puertos desde el host a la VM (por ejemplo, mapear host:8022 -> VM:22 para SSH).
		\item Se configuran en el \textbf{Virtual Network Editor}, parte de VMware Workstation Pro.
	\end{itemize}
\end{frame}

%----------------------------------
\subsection{15.2 Snapshots y Clones}
\begin{frame}{Snapshots}
	\begin{itemize}
		\item \textbf{Snapshot}: Captura del estado de la VM en un momento dado (sistema de archivos, memoria, configuración).
		\item Permite \textbf{restaurar} la VM a ese punto exacto.
		\item \textbf{Consejo}: Crear un snapshot antes de realizar cambios de sistema relevantes o actualizaciones arriesgadas.
	\end{itemize}
\end{frame}

\begin{frame}{Clones}
	\begin{itemize}
		\item \textbf{Clonado de VMs}:
			\begin{itemize}
				\item \textbf{Full Clone}: Copia completa e independiente de la VM original.
				\item \textbf{Linked Clone}: Comparte los discos base; ocupa menos espacio, pero depende de la VM “padre”.
			\end{itemize}
		\item \textbf{Uso común}: Crear plantillas y desplegar múltiples VMs rápidamente con diferentes propósitos.
	\end{itemize}
\end{frame}

%----------------------------------
\subsection{15.3 Cifrado de Máquinas Virtuales}
\begin{frame}{15.3 Cifrado en VMware}
	\begin{itemize}
		\item Disponible en \textbf{VMware Workstation Pro}.
		\item Cifra los archivos de la VM (discos virtuales, configuración), protegiendo contra acceso no autorizado.
		\item \textbf{Limitaciones}:
			\begin{itemize}
				\item Aumenta el consumo de CPU durante el cifrado/descifrado.
				\item Se necesita una contraseña o clave para iniciar la VM.
				\item Copias de seguridad pueden requerir la misma clave para restaurar.
			\end{itemize}
	\end{itemize}
\end{frame}

%----------------------------------
\subsection{15.4 GPU Passthrough (paso directo de GPU al invitado)}
\begin{frame}{15.4 GPU Passthrough}
	\begin{itemize}
		\item \textbf{Requisitos}:
			\begin{itemize}
				\item Soporte de \textbf{IOMMU} en la placa base y CPU (Intel VT-d / AMD-Vi).
				\item Firmware UEFI que permita asignar dispositivos PCIe a la VM.
			\end{itemize}
		\item \textbf{Limitaciones en hypervisores tipo 2}: No siempre está totalmente soportado, más común en ESXi o Proxmox.
		\item \textbf{Uso real}: Permite a la VM usar la GPU con controladores nativos (aplicaciones 3D, machine learning, etc.).
	\end{itemize}
\end{frame}

%----------------------------------
\subsection{15.5 Configuración de Disco}
\begin{frame}{Thin vs Thick, archivos partidos o en uno solo}
	\begin{itemize}
		\item \textbf{Thin Provision}: Crece dinámicamente, más eficiente al inicio.
		\item \textbf{Thick Provision}: Reserva todo el espacio de disco desde el principio, predecible en rendimiento.
		\item \textbf{Split disk} vs \textbf{single file}:
			\begin{itemize}
				\item \textbf{Split disk}: Útil en sistemas de archivos que no manejan bien ficheros muy grandes, o para facilitar backups parciales.
				\item \textbf{Single file}: Un solo \texttt{.vmdk}, más simple de gestionar, pero puede ser grande.
			\end{itemize}
	\end{itemize}
\end{frame}

%----------------------------------
\subsection{15.6 Hardware Version (virtual hardware compatibility)}
\begin{frame}{Hardware Version en VMware}
	\begin{itemize}
		\item \textbf{Hardware Version} define las características que la VM puede usar (ej. USB 3.1, NVMe, etc.).
		\item \textbf{v16, v17, etc.} según la versión de VMware.
		\item Se puede \textbf{downgradear} para compatibilidad con versiones anteriores de Workstation o ESXi, perdiendo funciones recientes.
	\end{itemize}
\end{frame}

%=====================================================================
% 16. Resolución de Problemas Frecuentes
%=====================================================================
\section{Resolución de Problemas Frecuentes}

%----------------------------------
\subsection{16.1 La VM no arranca (VT-x / AMD-V deshabilitado)}
\begin{frame}{VT-x / AMD-V deshabilitado}
	\begin{itemize}
		\item Revisar \textbf{BIOS/UEFI} para habilitar virtualización (“Intel VT-x”, “SVM” o similares).
		\item A veces, otra tecnología (Hyper-V en Windows) bloquea el acceso a VT-x.
		\item Mensajes típicos: “\textit{This host supports Intel VT-x, but it is disabled}” o “\textit{VMX/AMD-V not enabled}”.
	\end{itemize}
\end{frame}

%----------------------------------
\subsection{16.2 No se detecta la ISO}
\begin{frame}{No se detecta la ISO}
	\begin{itemize}
		\item Asegurarse de haber seleccionado el \textbf{archivo ISO} correcto en la configuración de la unidad de CD/DVD.
		\item Verificar que la opción \textbf{“Connect at power on”} esté marcada.
		\item Si la ISO está corrupta, la VM puede no arrancar. Revisa \textbf{checksums}.
	\end{itemize}
\end{frame}

%----------------------------------
\subsection{16.3 Red no disponible o ping no funciona}
\begin{frame}{Problemas de Red}
	\begin{itemize}
		\item Modo NAT: ¿Está el firewall del host bloqueando tráfico?
		\item Modo Bridged: Verificar que la red del host permita direcciones DHCP. A veces, hay que desactivar Wi-Fi bridging en Windows.
		\item \textbf{Host-Only}: No hay salida a Internet, solo comunicación con el host; confirmarlo si necesitas conectarte a la LAN o WAN.
	\end{itemize}
\end{frame}

%----------------------------------
\subsection{16.4 Pantalla pequeña o sin aceleración 3D}
\begin{frame}{Problemas Gráficos}
	\begin{itemize}
		\item Instalar \textbf{open-vm-tools-desktop} para habilitar la integración gráfica en Ubuntu Desktop.
		\item Revisa la configuración de video en la VM. Aumentar la memoria de video si es muy baja.
		\item Aceleración 3D requiere una GPU en el host y los drivers correctos. Si no está soportado, se usará renderizado por software.
	\end{itemize}
\end{frame}

%----------------------------------
\subsection{16.5 GRUB no se instala o no se ve}
\begin{frame}{Problemas con GRUB}
	\begin{itemize}
		\item Modo UEFI: ¿La partición EFI está creada y montada en \texttt{/boot/efi}?
		\item Modo BIOS: ¿Se seleccionó el disco correcto para instalar el cargador (ej. /dev/sda)?
		\item A veces puede ser necesario \textbf{reparar GRUB} con un disco live o usar la herramienta “Boot-Repair”.
	\end{itemize}
\end{frame}

%----------------------------------
\subsection{16.6 Problemas con Secure Boot}
\begin{frame}{Secure Boot}
	\begin{itemize}
		\item En Ubuntu con Secure Boot activo, los módulos de VMware (vmmon, vmnet) podrían requerir \textbf{firma digital}.
		\item Se puede firmar manualmente o \textbf{deshabilitar Secure Boot} temporalmente en la UEFI.
		\item Mensajes típicos: “\textit{Sign these kernel modules before using them in a Secure Boot environment}”.
	\end{itemize}
\end{frame}

%----------------------------------
\subsection{16.7 El ratón o teclado no responde (captura de entrada)}
\begin{frame}{Problemas de Ratón/Teclado}
	\begin{itemize}
		\item VMware “captura” el ratón/teclado cuando clicas dentro de la ventana de la VM.
		\item Salir con Ctrl+Alt para liberar el puntero.
		\item Si no responde, revisar \textbf{VMware Tools} instalados, o que no haya conflictos con otros drivers.
	\end{itemize}
\end{frame}

%----------------------------------
\subsection{16.8 Uso excesivo de CPU o RAM}
\begin{frame}{Optimización de Recursos}
	\begin{itemize}
		\item Revisa cuántos núcleos y cuánta RAM asignaste a la VM. ¿Estás sobreasignando?
		\item Uso excesivo de CPU puede indicar procesos dentro de la VM que consumen muchos recursos.
		\item Monitorea con herramientas en el host (\texttt{top}, \texttt{htop} en Linux) y en la VM, ajusta según sea necesario.
	\end{itemize}
\end{frame}

%=====================================================================
% 17. Glosario Extendido de Siglas y Términos
%=====================================================================
\section{Glosario Extendido de Siglas y Términos}

\begin{frame}{Glosario Extendido (I)}
	\begin{description}
		\item[BIOS] Basic Input/Output System
		\item[UEFI] Unified Extensible Firmware Interface
		\item[CSM] Compatibility Support Module (modo de compatibilidad BIOS en UEFI)
		\item[MBR] Master Boot Record
		\item[GPT] GUID Partition Table
	\end{description}
\end{frame}

\begin{frame}{Glosario Extendido (II)}
	\begin{description}
		\item[LVM] Logical Volume Manager
		\item[NVRAM] Non-Volatile Random-Access Memory (usada por UEFI para almacenar variables)
		\item[IOMMU] Input–Output Memory Management Unit (para passthrough de dispositivos)
		\item[VT-x/VT-d] Intel Virtualization Technology (para CPU y dispositivos)
		\item[AMD-V / AMD-Vi] Tecnologías equivalentes en AMD
	\end{description}
\end{frame}

\begin{frame}{Glosario Extendido (III)}
	\begin{description}
		\item[Host] Sistema operativo principal donde corre el hypervisor.
		\item[Guest] Sistema operativo invitado, corriendo dentro de la VM.
		\item[Snapshot] Captura del estado actual de la VM.
		\item[Passthrough] Asignar un dispositivo físico directamente a la VM.
		\item[ISO] Imagen de disco óptico (CD/DVD) utilizada para instalaciones.
	\end{description}
\end{frame}



\end{document}

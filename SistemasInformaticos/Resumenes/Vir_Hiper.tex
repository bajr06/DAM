\documentclass{beamer}

\usepackage[spanish]{babel}
\usepackage[utf8]{inputenc}
\usepackage[T1]{fontenc}
\usefonttheme{professionalfonts}
\usetheme{Madrid}
\usecolortheme{default}

\title{Virtualización con VMware y Ubuntu}
\subtitle{Parte 2/4}
\author{Víctor de Juan}
\date{\today}

\begin{document}

%----------------------------------
% Portada
%----------------------------------
\begin{frame}
	\titlepage
\end{frame}

%----------------------------------
% Índice de la Parte 2
%----------------------------------
\begin{frame}{Índice}
	\tableofcontents[
		hideallsubsections]
\end{frame}

%=====================================================================
% 6. Hypervisores: Tipo 1 y Tipo 2
%=====================================================================
\section{Hypervisores: Tipo 1 y Tipo 2}

%----------------------------------
\subsection{6.1 Ejemplos de Hypervisores Tipo 1 (Bare Metal)}
\begin{frame}{Hypervisores de Tipo 1 (Bare Metal) - Conceptos}
	\begin{itemize}
		\item \textbf{Definición:} Se instalan directamente sobre el hardware (no necesitan un sistema operativo anfitrión).
		\item Conocidos como \textit{“bare metal”} porque corren “a pelo” en la máquina física.
		\item Orientados a entornos de servidores y centros de datos por su alto rendimiento y capacidad de gestión.
	\end{itemize}
\end{frame}

\begin{frame}{Ejemplos de Hypervisores Tipo 1}
	\begin{itemize}
		\item \textbf{VMware ESXi:} Amplio uso empresarial, integra funciones de clustering y alta disponibilidad.
		\item \textbf{Microsoft Hyper-V (versión Server):} Incluido en Windows Server; gestión centralizada con herramientas de Microsoft.
		\item \textbf{Xen / Citrix Hypervisor:} Otro hypervisor popular en entornos de nube y virtualización empresarial.
		\item \textbf{KVM (en Linux):} A menudo considerado Tipo 1 cuando corre directamente con el kernel Linux como hypervisor.
	\end{itemize}
\end{frame}

%----------------------------------
\subsection{6.2 Ejemplos de Hypervisores Tipo 2 (Hosted)}
\begin{frame}{Hypervisores de Tipo 2 (Hosted) - Conceptos}
	\begin{itemize}
		\item \textbf{Definición:} Se ejecutan sobre un sistema operativo anfitrión (Windows, Linux o macOS).
		\item El hypervisor es una aplicación más, que “hospeda” las máquinas virtuales.
		\item Más sencillos de instalar y usar en equipos de escritorio o portátiles.
	\end{itemize}
\end{frame}

\begin{frame}{Ejemplos de Hypervisores Tipo 2}
	\begin{itemize}
		\item \textbf{VMware Workstation / Player}: Gran integración y estabilidad, ideal para entornos de desarrollo y pruebas.
		\item \textbf{Oracle VirtualBox}: Software libre y muy extendido para usuarios domésticos y laboratorios académicos.
		\item \textbf{Parallels Desktop (macOS)}: Orientado especialmente para virtualizar Windows y Linux en macOS.
		\item \textbf{Microsoft Hyper-V (versión Windows 10/11 Pro)}: Se considera Tipo 2 cuando corre “encima” de Windows.
	\end{itemize}
\end{frame}

%=====================================================================
% 7. VMware Workstation/Player: Descripción y Diferencias
%=====================================================================
\section{VMware Workstation/Player: Descripción y Diferencias}

%----------------------------------
\subsection{7.1 VMware Workstation Pro vs VMware Workstation Player}
\begin{frame}{Comparativa: Workstation Pro vs Workstation Player (I)}
	\begin{itemize}
		\item \textbf{Workstation Player (gratuito para uso personal)}:
			\begin{itemize}
				\item Menos funciones avanzadas (por ejemplo, no soporta crear múltiples snapshots o redes virtuales personalizadas complejas).
				\item Adecuado para usuarios que necesitan ejecutar una o pocas VMs para tareas básicas.
			\end{itemize}
		\item \textbf{Workstation Pro (licencia de pago)}:
			\begin{itemize}
				\item Soporta redes virtuales avanzadas, cifrado de VMs, clonaciones (linked/full) y múltiples snapshots.
				\item Ideal para desarrollo, testing y escenarios empresariales.
			\end{itemize}
	\end{itemize}
\end{frame}

%----------------------------------
\subsection{7.2 Licenciamiento y Usos}
\begin{frame}{7.2 Licenciamiento y Usos}
	\begin{itemize}
		\item \textbf{Uso personal y educativo}: Se suele optar por VMware Player si no se requieren las funcionalidades adicionales.
		\item \textbf{Uso comercial o laboral}: Generalmente se recomienda Workstation Pro, que incluye soporte oficial.
		\item \textbf{Descuentos académicos}: Algunas instituciones ofrecen licencias de VMware a personal y estudiantes.
	\end{itemize}
\end{frame}

%----------------------------------
\subsection{7.3 Compatibilidad (Windows, Linux, macOS)}
\begin{frame}{7.3 Compatibilidad}
	\begin{itemize}
		\item \textbf{Windows y Linux}: Workstation Pro/Player está disponible para ambos. Requisitos de sistema varían.
		\item \textbf{macOS}: No existe Workstation como tal; en Mac se utiliza \textbf{VMware Fusion}.  
		\item \textbf{Rendimiento}: Similar en ambos, pero depende de drivers y kernel. A veces la versión de Linux puede requerir recompilar módulos.
	\end{itemize}
\end{frame}

%=====================================================================
% 8. Instalación de VMware Workstation/Player en Linux (Ubuntu)
%=====================================================================
\section{Instalación de VMware Workstation/Player en Linux (Ubuntu)}

%----------------------------------
\subsection{8.1 Descargar el instalador}
\begin{frame}{8.1 Descarga el instalador}
	\begin{itemize}
		\item \textbf{Mucho ánimo y mucha suerte}
	\end{itemize}
\end{frame}


\begin{frame}{8.2 Instalación (II)}
	\begin{itemize}
		\item Aparecerá un asistente gráfico o una instalación en la terminal (dependiendo la versión).
		\item Aceptar la licencia de VMware y continuar.
	\end{itemize}
\end{frame}

%----------------------------------

\begin{frame}{8.3 Módulos de Kernel y requerimientos}
	\begin{block}{Nota sobre Secure Boot}
		En sistemas con UEFI y Secure Boot habilitado, puede requerirse \textbf{firmar} los módulos para que puedan cargarse.  
		O desactivar temporalmente Secure Boot para facilitar la instalación.
	\end{block}
	\begin{itemize}
		\item Tras la instalación, VMware puede solicitar recompilar los módulos al actualizar el kernel
		\item Revisar logs si aparece algún error.
	\end{itemize}
\end{frame}

%----------------------------------
\subsection{8.4 Posibles problemas (dependencias, permisos, etc.)}
\begin{frame}{8.4 Problemas comunes}
	\begin{itemize}
	\item \textbf{Dependencias no satisfechas:} Asegurarse de tener instalados los paquetes de compilación.
	\item \textbf{Permisos:} Ejecutar el instalador como usuario root o con \texttt{sudo}.
	\item \textbf{Secure Boot:} Puede impedir que se carguen los módulos no firmados.
	\item \textbf{Conflicto con Hyper-V en Windows (solo si usas dual-boot o WSL2)}:
		\begin{itemize}
			\item Asegúrate de que Hyper-V esté deshabilitado al usar VMware en Windows.  
			\item En Linux no suele haber conflicto con WSL2, pero en dual-boot sí.
		\end{itemize}
	\end{itemize}
\end{frame}

%=====================================================================
% 9. Creación de una Máquina Virtual en VMware
%=====================================================================
\section{Creación de una Máquina Virtual en VMware}

%----------------------------------
\subsection{9.1 Configuración Típica vs Configuración Avanzada}
\begin{frame}{9.1 Configuración Típica vs Avanzada}
	\begin{itemize}
		\item \textbf{Configuración Típica (Typical):} Asistente rápido, la mayoría de valores se autoconfiguran (memoria, disco, etc.).
		\item \textbf{Configuración Avanzada (Custom):} Permite ajustar detalles como:
			\begin{itemize}
				\item Tipo de compatibilidad de hardware (Workstation 17, 16, etc.).
				\item Tipo de firmware (BIOS/UEFI).
				\item Formato de disco (SCSI, NVMe, SATA).
			\end{itemize}
		\item Recomendado el modo \textbf{Custom} para mayor control o escenarios especiales.
	\end{itemize}
\end{frame}

%----------------------------------
\subsection{9.2 Compatibilidad de Hardware (Hardware Version)}
\begin{frame}{9.2 Compatibilidad de Hardware (Hardware Version)}
	\begin{itemize}
		\item Cada versión de Workstation (o ESXi) introduce una “\textbf{Hardware Version}”.
		\item Determina las características soportadas: PCI passthrough, NVMe, USB 3.0, etc.
		\item Ejemplo: \textbf{Hardware Version 17} en VMware Workstation 17, \textbf{Hardware Version 16} en VMware Workstation 16.
		\item \textbf{Retrocompatibilidad:} Si necesitas que la VM funcione en una versión anterior, puedes elegir una versión de hardware inferior.
	\end{itemize}
\end{frame}

%----------------------------------
\subsection{9.3 Selección de Firmware (BIOS o UEFI en la VM)}
\begin{frame}{9.3 Selección de Firmware}
	\begin{itemize}
		\item En la configuración de la VM se puede elegir \textbf{BIOS} o \textbf{UEFI}.
		\item \textbf{UEFI} es más moderno y soporta arranque EFI, Secure Boot (beta en algunos VMware), discos GPT, etc.
		\item \textbf{BIOS} es adecuado para sistemas legacy o si el invitado no maneja UEFI correctamente.
	\end{itemize}
\end{frame}

%----------------------------------
\subsection{9.4 Tipos de Disco Virtual (VMDK, otros formatos)}
\begin{frame}{9.4 Tipos de Disco Virtual}
	\begin{itemize}
		\item \textbf{VMDK (formato nativo de VMware)}: Recomendado para su ecosistema.
		\item \textbf{Otros formatos}: VHD (Virtual PC / Hyper-V), VDI (VirtualBox), etc.
		\item Normalmente, VMware trabaja con VMDK, aunque ofrece herramientas para convertir a otros formatos (\textbf{vmware-vdiskmanager}).
		\item Elegir \textit{SCSI}, \textit{SATA} o \textit{NVMe} como controlador según la versión de hardware y necesidades de rendimiento.
	\end{itemize}
\end{frame}

%----------------------------------
\subsection{9.5 Thin Provision vs Thick Provision}
\begin{frame}{9.5 Thin vs Thick Provision
	\begin{itemize}
		\item \textbf{Thin Provision:} El archivo de disco virtual \texttt{.vmdk} crece a medida que se van usando los datos.  
			\begin{itemize}
				\item Ahorra espacio inicialmente.
				\item Puede fragmentarse más o tener impacto en rendimiento al crecer.
			\end{itemize}
		\item \textbf{Thick Provision:} Asigna todo el espacio en disco desde un inicio.
			\begin{itemize}
				\item Mayor consumo de espacio desde el principio.
				\item Más predecible en rendimiento.
			\end{itemize}
	\end{itemize}
\end{frame}

%----------------------------------
\subsection{9.6 Opciones de CPU y Memoria (Ballooning, Núcleos, etc.)}
\begin{frame}{9.6 CPU y Memoria}
	\begin{itemize}
		\item \textbf{Asignación de núcleos/hilos:} Depende de la CPU física, no conviene asignar más núcleos de los que realmente se tienen.
		\item \textbf{Ballooning}: Mecanismo de VMware para ajustar dinámicamente la memoria asignada a la VM si el host la necesita.
		\item \textbf{Overcommit de memoria}: Se puede asignar más memoria total a las VMs que la RAM física, pero podría haber swapping.
		\item Recomendación: Ajustar CPU y RAM de acuerdo a las necesidades reales del SO invitado.
	\end{itemize}
\end{frame}

%----------------------------------
\subsection{9.7 Configuración de Red (Bridged, NAT, Host-Only, redes personalizadas)}
\begin{frame}{9.7 Configuración de Red}
	\begin{itemize}
		\item \textbf{Bridged (vmnet0)}: La VM comparte la red física, obtiene IP del router (o DHCP) como si fuera otro equipo en la LAN.
		\item \textbf{NAT (vmnet8)}: La VM utiliza la IP del host como pasarela para salir a internet; IP interna en la red virtual.
		\item \textbf{Host-Only (vmnet1)}: Solo se comunica con el host y otras VMs host-only, sin salida a internet (ideal para laboratorios internos).
		\item \textbf{Redes personalizadas}: En Workstation Pro se pueden crear subredes virtuales, VLAN trunking, etc.
	\end{itemize}
\end{frame}

%----------------------------------
\subsection{9.8 Configuración de USB y otros periféricos (passthrough)}
\begin{frame}{9.8 Passthrough de USB y periféricos}
	\begin{itemize}
		\item Conecta un dispositivo USB directamente a la VM (ej: pendrives, discos externos, impresoras).
		\item Posible conflicto si el host también quiere usar el dispositivo simultáneamente.
		\item En VMware, se elige la VM como “destino” del dispositivo en la interfaz.
		\item \textbf{Requisito:} Instalar drivers dentro de la VM si el SO invitado los requiere.
	\end{itemize}
\end{frame}

%----------------------------------
\subsection{9.9 Ajustes de Gráficos (Aceleración 3D)}
\begin{frame}{9.9 Ajustes de Gráficos}
	\begin{itemize}
		\item \textbf{Aceleración 3D}: VMware ofrece un driver gráfico paravirtualizado.
		\item Recomendado habilitarla para entornos de escritorio (Unity, GNOME, KDE) y rendimiento de video.
		\item No es lo mismo que \textbf{GPU passthrough}, que requiere acceso directo al hardware (limitado en Workstation).
		\item \textbf{Configurar la VRAM}: Ajustar la memoria de video según las necesidades de la aplicación.
	\end{itemize}
\end{frame}

%=====================================================================
% 10. Descarga y Preparación de la ISO de Ubuntu
%=====================================================================
\section{Descarga y Preparación de la ISO de Ubuntu}

%----------------------------------
\subsection{10.1 Tipos de Ubuntu (Desktop, Server, Flavors)}
\begin{frame}{10.1 Tipos de Ubuntu}
	\begin{itemize}
		\item \textbf{Ubuntu Desktop}: Incluye entorno gráfico (GNOME), aplicaciones de oficina, navegador, etc.
		\item \textbf{Ubuntu Server}: Más ligero, sin entorno gráfico por defecto, orientado a servidores y servicios.
		\item \textbf{Flavors}: Kubuntu (KDE), Xubuntu (Xfce), Lubuntu (LXQt), Ubuntu MATE, etc.  
			\begin{itemize}
				\item Mismas bases de Ubuntu, diferentes entornos de escritorio.
			\end{itemize}
		\item Elegir según recursos y propósito (escritorio o servidor).
	\end{itemize}
\end{frame}

%----------------------------------
\subsection{10.2 Verificación de la ISO (Checksum)}
\begin{frame}{10.2 Verificación de la ISO}
	\begin{itemize}
		\item Tras descargar la ISO de la web oficial, se recomienda comprobar la \textbf{integridad} con \texttt{sha256sum} o \texttt{md5sum}.
		\item \textbf{Ejemplo de uso:}  
			\begin{itemize}
				\item \texttt{sha256sum ubuntu-22.04-desktop-amd64.iso}
				\item Comparar el resultado con el valor en la página oficial.
			\end{itemize}
		\item Garantiza que el archivo no se ha corrompido ni modificado de forma maliciosa.
	\end{itemize}
\end{frame}



\end{document}

\documentclass{article}

\usepackage{amsmath}
\usepackage{amssymb}

\title{04.4 Ejercicios de Álgebra Relacional}
\author{Bryan Andreu Jiménez Rojas}
\date{Febrero 2025}

\begin{document}
\maketitle

\textnormal{1. Obtener el nombre de las papelerías que se encuentran en Madrid}
\begin{equation*}
	\begin{flalign*}
		& \Pi_{Nombre} \left(\sigma_{Ciudad = 'Madrid'} \left(\text{Papeleras}\right)\right)
	\end{flalign*}
\end{equation*}

\textnormal{2. Obtener el nombre de todas las editoriales que se encuentren en Barcelona}
\begin{equation*}
	\begin{flalign*}
		& \Pi_{Nombre} \left(\sigma_{Ciudad = 'Barcelona'} \left(\text{Editoriales}\right)\right) \\
		% Invención de Beatriz.
		& T1 \leftarrow \Pi_{Ciudades} \left(\text{Editoriales}\right) \\
		& T2 \leftarrow \Pi_{Ciudades} \left(\text{Papeleras}\right) \\
		& \Pi_{Ciudades} \left(T1 \cap T2\right)
	\end{flalign*}
\end{equation*}

\textnormal{3. Obtener todos los autores que han publicado libros en el 2010}
\begin{equation*}
	\begin{flalign*}
		& \Pi_{Autores} \left(\right) % TODO
	\end{flalign*}
\end{equation*}

\textnormal{4. Obtener el nombre de los libros que tiene ha publicado la editorial rama}
\begin{equation*}
	\begin{flalign*}
		& T1 \leftarrow \Pi_{Código\_Editorial} \left(\sigma_{Nombre = 'Rama'} \left(\text{Editorial}\right)\right) \\
		& T2 \leftarrow \Pi_{Identificador\_Libro} \left(T1 \bowtie ELP.\right) \\
		& \text{T1.Código\_Editorial = ELP.Código\_Editorial} \\
		& \Pi_{Titulo} \left(T2 \bowtie \text{Libro}\right) \\
		& \text{T2.Identificador\_Libro = Libros.Identificador\_Libro}
	\end{flalign*}
\end{equation*}

\textnormal{5. Obtener todos los autores de las editoriales que están afincadas en Sevilla}
\begin{equation*}
	\begin{flalign*}
		& T1 \leftarrow \Pi_{Nombre} \left(\sigma_{Ciudad = 'Sevilla'} \left(\text{Editoriales}\right)\right) \\
		& T2 \leftarrow \Pi_{Identificador\_Libros} \left(T1 \bowtie ELP \right) \\
		& \text{T1.Código\_Editorial = ELP.Código\_Editorial} \\\\
		& \Pi_{Autor} \left(T2 \bowtie Libros\right)\\
		& \text{T2.Identificador\_Libro = Libros.Identificador\_Libro}
	\end{flalign*}
\end{equation*}

\textnormal{6. Obtener los nombres de las papelerías que han sido abastecidas por alguna editorial de Madrid}
\begin{equation*}
	\begin{flalign*}
		& T1 \leftarrow \Pi_{Código\_Editorial} \left(\sigma_{Ciudad = 'Madrid'}\right) \\
		& T2 \leftarrow \left(T1 \bowtie ELP\right) \\
		& \text{T1.Código\_Editorial = ELP.Código\_Editorial}
		\\\\
		& \Pi_{Nombre} \left(T2 \bowtie Papeleria\right) \\
		& \text{T2.Código\_Identificador\_Papeleria = Papeleria.Identificador\_Papeleria}
		\\\\\\
		% Invención Beatriz
		& T1 \leftarrow \Pi_{Precio} \left(\sigma_{Titulo = 'A} \left(\text{Libros}\right)\right) \\
		& Libros \chi T1 \\
		& \Pi_{Nombre} \left(\sigma_{Libros.Precios = T1.Precio}\left(Libros \chi \T1\right)\right)
	\end{flalign*}
\end{equation*}

\textnormal{7. Obtener los nombres de los libros que tienen las papelerías que están en Cáceres}
\begin{equation*}
	\begin{flalign*}
		& T1 \leftarrow \Pi_{Identificador\_Papeleria} \left(\sigma_{Ciudad = "Cáceres"} \left(\text{Papeleria}\right)\right) \\
		& \Pi_{Identificador\_Libro} \left(T1 \bowtie ELP\right) \\
		& T2 \leftarrow \Pi_{Identificador\_Libro}\text{T1.Identificador\_Papeleria = ELP.Identificador\_Papeleria}
		\\\\
		& \Pi_{Nombre} \left(T2 \bowtie Libro\right) \\
		& \text{T2.Identificador\_Libro = Libro.Identificador\_Libro}
	\end{flalign*}
\end{equation*}

\textnormal{}
\begin{equation*}
	\begin{flalign*}
		& T1 \leftarrow \Pi_{Identificador\_Papeleria}\left(\sigma_{Nombre = 'P1'}\left(\text{Papeleria}\right)\right) \\
		& T2 \leftarrow \Pi_{Identificador\_Papeleria}\left(\sigma_{Nombre = 'P3'}\left(\text{Papeleria}\right)\right) \\
		& T3 \leftarrow \Pi_{Código\_Editorial}\left(T1 \bowtie ELP\right) \\
		& \text{T1.Identificador\_Papeleria = ELP.Indentificador\_Papeleria} \\
		& T2 \cap \T4
		& T4 \leftarrow \Pi_{Código\_Editorial} \left(T2 \bowtie ELP\right) \\
		& \text{T2.Identificador\_Papeleria = ELP.Identificador\_Papeleria}
	\end{flalign*}
\end{equation*}

\end{document}

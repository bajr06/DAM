\documentclass{beamer}

\usepackage[spanish]{babel}
\usepackage[utf8]{inputenc}
\usepackage[T1]{fontenc}
\usefonttheme{professionalfonts}
\usetheme{Madrid}
\usecolortheme{default}

\title{Virtualización de Sistemas Operativos}
\subtitle{Parte 1/4}
\author{Víctor de Juan}
\date{\today}

\begin{document}

%----------------------------------
% Portada
%----------------------------------
\begin{frame}
	\titlepage
\end{frame}

%----------------------------------
% Índice de la Presentación
%----------------------------------
\begin{frame}{Índice - Parte 1}
	\tableofcontents[hideallsubsections]
\end{frame}

%=====================================================================
% 1. Conceptos Fundamentales de Virtualización
%=====================================================================
\section{Conceptos Fundamentales de Virtualización}

%----------------------------------
\subsection{Definición de Virtualización}
\begin{frame}{1.1 Definición de Virtualización (I)}
	\begin{itemize}
		\item \textbf{Virtualización:} Proceso de crear una representación virtual de un recurso físico, como un servidor, sistema operativo, dispositivo de almacenamiento o red.
		\item \textbf{Objetivo principal:} Separar el software del hardware subyacente.
		\item \textbf{¿Para qué?} Permite ejecutar múltiples sistemas operativos en la misma máquina física de forma simultánea.
	\end{itemize}
\end{frame}

\begin{frame}{1.1 Definición de Virtualización (II)}
	\begin{itemize}
		\item Cada entorno virtual se denomina \textbf{Máquina Virtual} (VM).  
		\item El sistema operativo donde se ejecutan las VMs se llama \textbf{Host}, y los sistemas virtualizados son los \textbf{Guests} o “invitados”.
		\item Virtualización vs Emulación: la virtualización aprovecha recursos reales con ayuda de la CPU (VT-x/AMD-V), mientras que la emulación reproduce completamente otra arquitectura (más lento).
	\end{itemize}
	\vspace{0.3cm}
	\textbf{Ejemplo:} Ejecutar Windows y Ubuntu en la misma computadora al mismo tiempo, pero aislados.
\end{frame}

%----------------------------------
\subsection{Historia y Evolución de la Virtualización}
\begin{frame}{1.2 Historia y Evolución (I)}
	\begin{itemize}
		\item \textbf{Décadas de 1960-1970:} IBM desarrolla técnicas de virtualización para compartir recursos en grandes mainframes.
		\item \textbf{Década de 1990:} VMware introduce la virtualización en arquitecturas x86, permitiendo correr varios sistemas operativos de escritorio/servidor.
		\item \textbf{Años 2000 en adelante:} Crecimiento exponencial en centros de datos y surgimiento de la computación en la nube.
	\end{itemize}
\end{frame}

\begin{frame}{1.2 Historia y Evolución (II)}
	\begin{itemize}
		\item \textbf{Hypervisores de tipo 1:} Ej. VMware ESXi, orientados a entornos de servidor.  
		\item \textbf{Hypervisores de tipo 2:} Ej. VMware Workstation, VirtualBox, orientados a equipos de escritorio.
		\item \textbf{Actualidad:} Virtualización y contenedores (Docker, Kubernetes) son pilares fundamentales de la computación moderna.
	\end{itemize}
	\vspace{0.3cm}
	\textbf{Importancia:} Permite un uso más eficiente de los recursos y acelera el despliegue de nuevos servicios.
\end{frame}

%----------------------------------
\subsection{Beneficios Principales}
\begin{frame}{1.3 Beneficios Principales (I)}
	\begin{itemize}
		\item \textbf{Aislamiento:} Problemas en una VM no afectan otras VMs ni al sistema principal.
		\item \textbf{Consolidación:} Varios servicios o sistemas corriendo en un solo equipo físico (menos hardware).
		\item \textbf{Flexibilidad:} Permite pruebas de distintas configuraciones sin comprometer la máquina real.
	\end{itemize}
\end{frame}

\begin{frame}{1.3 Beneficios Principales (II)}
	\begin{itemize}
		\item \textbf{Despliegue rápido:} Clonar o crear nuevas VMs es mucho más ágil que instalar en hardware físico.
		\item \textbf{Recuperación ante desastres:} Con snapshots y copias de seguridad es fácil revertir o restaurar.
		\item \textbf{Ahorro de costes:} Menos servidores físicos, reducción en consumo de energía y espacio.
	\end{itemize}
\end{frame}

%----------------------------------
\subsection{Desventajas o Desafíos}
\begin{frame}{1.4 Desventajas y Desafíos (I)}
	\begin{itemize}
		\item \textbf{Sobrecarga de hardware:} El hypervisor consume recursos y cada VM también, por lo que puede haber decremento en el rendimiento.
		\item \textbf{Complejidad:} Requiere conocimientos de virtualización, redes virtuales y licenciamiento.
		\item \textbf{Licenciamiento de software:} Cada VM puede requerir licencias adicionales para el sistema operativo y las aplicaciones.
	\end{itemize}
\end{frame}

\begin{frame}{1.4 Desventajas y Desafíos (II)}
	\begin{itemize}
		\item \textbf{Limitaciones físicas:} Si el host se daña, se pierden todas las VMs en él (a menos que haya alta disponibilidad).
		\item \textbf{Seguridad:} Aunque aísla las VMs, si el hypervisor es comprometido, existe un riesgo mayor para todas las VMs.
		\item \textbf{Planificación de recursos:} Difícil optimizar CPU, RAM y disco cuando se ejecutan muchas VMs a la vez.
	\end{itemize}
\end{frame}

%=====================================================================
% 2. Requisitos Previos de Hardware y Software
%=====================================================================
\section{Requisitos Previos de Hardware y Software}

%----------------------------------
\subsection{Compatibilidad de CPU}
\begin{frame}{2.1 Compatibilidad de CPU (I)}
	\begin{itemize}
		\item \textbf{Intel VT-x (Virtualization Technology)} o \textbf{AMD-V:} Extensiones de CPU que permiten la ejecución de VMs con mayor rendimiento.
		\item Verificar en la \textbf{BIOS/UEFI} que estén habilitadas. A veces figuran como “Virtualization Technology” o “SVM Mode”.
	\end{itemize}
\end{frame}

\begin{frame}{2.1 Compatibilidad de CPU (II)}
	\begin{itemize}
		\item \textbf{Intel VT-d / AMD-Vi (IOMMU):} Funcionalidad adicional para asignar dispositivos físicos directamente a VMs (passthrough).
		\item Importante en entornos donde se necesite acceder a hardware especializado, como tarjetas gráficas dedicadas.
	\end{itemize}
	\textbf{Nota:} No todas las CPU de gama baja incluyen estas características; es clave revisarlo antes de intentar la virtualización.
\end{frame}

%----------------------------------
\subsection{Compatibilidad de la Placa Base}
\begin{frame}{2.2 Compatibilidad de la Placa Base}
	\begin{itemize}
		\item \textbf{BIOS/UEFI actualizada:} Verificar en la web del fabricante para ver si hay soporte mejorado de virtualización.
		\item El chipset debe soportar las funciones de virtualización: \textbf{Intel Q series} (Q35, etc.) o \textbf{AMD 970, 990, etc.}
		\item Opciones en el menú de configuración: a veces llamadas \textbf{“Intel Virtualization Tech”} o \textbf{“SVM Support”}.
	\end{itemize}
\end{frame}

%----------------------------------
\subsection{Memoria RAM recomendada}
\begin{frame}{2.3 Memoria RAM recomendada}
	\begin{itemize}
		\item \textbf{8 GB de RAM mínimo} en el host para laboratorios sencillos (1-2 VMs ligeras).
		\item \textbf{16 GB o más:} recomendado para ejecutar varias VMs simultáneamente (por ejemplo, un entorno de desarrollo).
		\item Tener en cuenta que la RAM asignada a la VM deja menos disponible para el sistema host.
	\end{itemize}
\end{frame}

%----------------------------------
\subsection{Espacio en Disco y tipo de almacenamiento}
\begin{frame}{2.4 Espacio en Disco y Tipo de Almacenamiento}
	\begin{itemize}
		\item \textbf{Disco SSD:} Lectura y escritura más rápidas, reduce el tiempo de arranque de VMs.
		\item \textbf{Capacidad:} Una instalación de Ubuntu Desktop puede requerir entre 10 y 20 GB, más los datos que se generen.
		\item \textbf{Planificar el número de VMs:} Con 2-3 VMs simultáneas, se pueden necesitar 80-100 GB (o más) dependiendo de las aplicaciones.
	\end{itemize}
\end{frame}

%----------------------------------
\subsection{Requisitos de software del sistema anfitrión}
\begin{frame}{2.5 Requisitos de software del Host (I)}
	\begin{itemize}
		\item \textbf{Sistema Operativo Host:} Windows (10/11), Linux (variantes como Ubuntu, Fedora), macOS (en algunos casos).
		\item \textbf{VMware Workstation/Player o Hypervisor elegido:} Descargar la versión adecuada para tu SO host.
		\item Controladores de hardware actualizados para un mejor rendimiento y estabilidad.
	\end{itemize}
\end{frame}

\begin{frame}{2.5 Requisitos de software del Host (II)}
	\begin{itemize}
		\item \textbf{Herramientas de desarrollo (opcional en Linux):} Paquetes \texttt{build-essential}, \texttt{gcc}, \texttt{linux-headers}, etc., si el hypervisor compila módulos.
		\item \textbf{Herramientas de red:} \texttt{net-tools}, \texttt{iproute2}, etc., para verificar conectividad y redes virtuales.
		\item \textbf{Evitar conflictos:} entre software de virtualización
	\end{itemize}
\end{frame}

%----------------------------------
\subsection{Herramientas de diagnóstico}
\begin{frame}{2.6 Herramientas de diagnóstico}
	\begin{itemize}
		\item \texttt{lscpu}: Muestra información de la CPU y si soporta virtualización.
		\item \texttt{dmidecode}: Información del BIOS/UEFI y del sistema.
		\item \texttt{cpuinfo}: Dependiendo del SO, revisar /proc/cpuinfo en Linux.
		\item \textbf{Diagnóstico en Windows:} \texttt{msinfo32}, o en “Información del sistema”.
	\end{itemize}
\end{frame}

%=====================================================================
% 3. Firmware del Equipo: BIOS vs UEFI
%=====================================================================
\section{Firmware del Equipo: BIOS vs UEFI}

%----------------------------------
\subsection{BIOS (Basic Input/Output System)}
\begin{frame}{3.1 BIOS (I)}
	\begin{itemize}
		\item Sistema tradicional utilizado en equipos más antiguos.
		\item \textbf{Función principal:} Inicializar y comprobar los componentes de hardware (POST) y cargar el sistema operativo.
		\item Limite de 2 TB cuando se usa MBR como tabla de particiones.
	\end{itemize}
\end{frame}

\begin{frame}{3.1 BIOS (II)}
	\begin{itemize}
		\item \textbf{Configuración:} Se accede generalmente con teclas como F2, Del, F10 al encender el equipo.
		\item Se elige el \textbf{orden de arranque} (CD, USB, disco duro).
		\item \textbf{Opciones de virtualización:} Activar VT-x/AMD-V puede aparecer como “Intel Virtualization” o “SVM”.
	\end{itemize}
\end{frame}

%----------------------------------
\subsection{UEFI (Unified Extensible Firmware Interface)}
\begin{frame}{3.2 UEFI (I)}
	\begin{itemize}
		\item \textbf{Evolución de BIOS}: Soporta discos de gran tamaño (>2 TB) y tiene una interfaz más moderna.
		\item \textbf{ESP (EFI System Partition)}: Partición FAT32 donde se almacenan los archivos de arranque (bootloaders).
		\item Soporta \textbf{gráficos y ratón} en el menú de configuración, lo que facilita su uso.
	\end{itemize}
\end{frame}

\begin{frame}{3.2 UEFI (II)}
	\begin{itemize}
		\item \textbf{Variables UEFI}: Almacenan ajustes en NVRAM (memoria no volátil).
		\item Ofrece \textbf{Secure Boot}, funcionalidad de seguridad que solo arranca software firmado.
		\item \textbf{Mayor flexibilidad}: No se limita a 4 particiones primarias como MBR, y puede tener múltiples entradas de arranque.
	\end{itemize}
\end{frame}

%----------------------------------
\subsection{Modos de compatibilidad (CSM)}
\begin{frame}{3.3 Modos de compatibilidad (CSM)}
	\begin{itemize}
		\item \textbf{CSM (Compatibility Support Module)}: Permite a la UEFI comportarse como BIOS Legacy.
		\item Útil cuando un sistema operativo o un disco no son compatibles con arranque en modo UEFI.
		\item Puede requerirse desactivar \textbf{Secure Boot} para usar CSM.
	\end{itemize}
\end{frame}

%=====================================================================
% 4. Tablas de Particiones: MBR vs GPT
%=====================================================================
\section{Tablas de Particiones: MBR vs GPT}

%----------------------------------
\subsection{Estructura interna de MBR y sus limitaciones}
\begin{frame}{4.1 Estructura de MBR (I)}
	\begin{itemize}
		\item \textbf{Master Boot Record (MBR):} Se ubica en el primer sector (512 bytes) del disco duro.
		\item Contiene \textbf{código de arranque} (bootloader primario) + tabla de particiones.
		\item Sólo soporta \textbf{4 particiones primarias} o 3 primarias + 1 extendida.
	\end{itemize}
\end{frame}

\begin{frame}{4.1 Estructura de MBR (II)}
	\begin{itemize}
		\item Límite de \textbf{2 TB} para el disco, lo que resulta insuficiente para servidores modernos.
		\item \textbf{Sector de arranque}: Almacena el programa que inicia el sistema operativo (ej. primera etapa de GRUB).
		\item Si se corrompe el MBR, el disco puede dejar de ser arrancable.
	\end{itemize}
\end{frame}

%----------------------------------
\subsection{GPT y ventajas sobre MBR}
\begin{frame}{4.2 GPT (I)}
	\begin{itemize}
		\item \textbf{GUID Partition Table (GPT)}: Usa identificadores únicos globales (GUID) para cada partición.
		\item Soporta \textbf{discos muy grandes}, más allá de los 2 TB.
		\item \textbf{Mayor número de particiones}: Sin la limitación de sólo 4 como en MBR.
	\end{itemize}
\end{frame}

\begin{frame}{4.2 GPT (II)}
	\begin{itemize}
		\item \textbf{Tabla de respaldo}: GPT guarda una copia al final del disco, facilitando la recuperación en caso de corrupción.
		\item Requerido en la mayoría de sistemas UEFI para el arranque nativo.
		\item Es la forma recomendada para discos modernos, especialmente en SSDs grandes.
	\end{itemize}
\end{frame}

%----------------------------------
\subsection{Tipos de Particiones (Primaria, Extendida, Lógica)}
\begin{frame}{4.3 Tipos de Particiones}
	\begin{itemize}
		\item \textbf{Partición Primaria:} Directamente arrancable; en MBR, solo 4 primarias como máximo.
		\item \textbf{Extendida:} Estructura contenedora para crear \textbf{particiones lógicas} adicionales.
		\item \textbf{Particiones Lógicas:} Se ubican dentro de la extendida; util para subdividir más de 4 particiones en MBR.
		\item En GPT no existe el concepto de primaria vs. extendida/lógica, todas las particiones son similares.
	\end{itemize}
\end{frame}

%----------------------------------
\subsection{Particiones en modo UEFI (EFI System Partition)}
\begin{frame}{4.4 EFI System Partition (ESP)}
	\begin{itemize}
		\item Partición necesaria en equipos con UEFI donde se instalan los \textbf{bootloaders} (por ej. GRUB EFI).
		\item \textbf{Formato FAT32}, suele tener entre 100 y 300 MB.
		\item \textbf{Ruta típica}: \texttt{/boot/efi} en Linux, con una carpeta para cada sistema operativo.
		\item Permite arranques múltiples (dual-boot) de forma más sencilla que MBR.
	\end{itemize}
\end{frame}

%=====================================================================
% 5. Tipos de Virtualización
%=====================================================================
\section{Tipos de Virtualización}

%----------------------------------
\subsection{Virtualización Completa}
\begin{frame}{5.1 Virtualización Completa (I)}
	\begin{itemize}
		\item El hypervisor emula o abstrae todo el hardware: CPU, memoria, controladores de red, etc. 
			\subitem El disco duro podría no virtualizarse en hipervisores de tipo 1 en un entorno de servidor.
		\item El sistema operativo invitado se instala “como si” estuviera en un hardware físico real.
		\item Ejemplos: \textbf{VMware Workstation}, \textbf{VirtualBox} (modo completo), QEMU (con aceleración).
	\end{itemize}
\end{frame}

\begin{frame}{5.1 Virtualización Completa (II)}
	\begin{itemize}
		\item \textbf{Ventajas:} Compatibilidad amplia, cualquier SO que corra en esa arquitectura.
		\item \textbf{Desventajas:} Mayor overhead comparado con otras técnicas (paravirtualización).
		\item Útil para laboratorios donde se requiera probar diferentes sistemas operativos sin modificaciones.
	\end{itemize}
\end{frame}

%----------------------------------
\subsection{Paravirtualización}
\begin{frame}{5.2 Paravirtualización (I)}
	\begin{itemize}
		\item El \textbf{SO invitado} sabe que está virtualizado y utiliza drivers paravirtualizados para ciertas operaciones. 
			\begin{itemize}
				\item El SO invitado debe ser modificado para sustituir ciertas instrucciones por llamadas al hipervisor (a través de una API).        
			\end{itemize}
		\item Reduce la emulación de hardware, lo que se traduce en \textbf{mejor rendimiento}.
		\item Caso típico: \textbf{Xen}, \textbf{KVM} con virtio drivers en Linux.
	\end{itemize}
\end{frame}

\begin{frame}{5.2 Paravirtualización (II)}
	\begin{itemize}
		\item \textbf{Requiere cooperación}: El kernel invitado debe tener el soporte paravirtualizado.
		\item \textbf{Menor overhead}: Aprovecha instruccciones específicas y llamadas directas al hypervisor.
		\item Ideal para entornos de servidor donde se controlan las imágenes del sistema operativo.
		\item \textbf{wsl2} está paravirtualizado utilizando un hipervisor de tipo 2 (Hyper-V porque corre sobre Windows 10/11).
	\end{itemize}
\end{frame}

%----------------------------------
\subsection{Virtualización a Nivel de SO (Contenedores)}
\begin{frame}{5.3 Virtualización a Nivel de SO (I)}
	\begin{itemize}
		\item Se basa en compartir el \textbf{mismo kernel} del host con múltiples entornos aislados.
		\item Ejemplos: \textbf{Docker}, \textbf{LXC}, \textbf{Podman}.
		\item \textbf{Ligereza:} Consumimos menos recursos que una máquina virtual completa.
	\end{itemize}
\end{frame}

\begin{frame}{5.3 Virtualización a Nivel de SO (II)}
	\begin{itemize}
		\item \textbf{Limitación:} Solo puede virtualizar sistemas que usen el mismo tipo de kernel (por ejemplo, contenedores Linux sobre host Linux).
		\item Muy popular en \textbf{DevOps} y despliegue continuo.
		\item Combina bien con orquestadores como \textbf{Kubernetes} para escalabilidad masiva.
	\end{itemize}
\end{frame}

%----------------------------------
\subsection{Emulación}
\begin{frame}{5.4 Emulación}
	\begin{itemize}
		\item El software \textbf{imita por completo} una arquitectura distinta a la del hardware real (ej. emular ARM sobre x86).
		\item Mucho más \textbf{lento} que la virtualización, pero posibilita compatibilidad total con SO diseñados para otra CPU.
		\item Ejemplo: \textbf{QEMU} emulando plataformas embebidas, consolas de videojuegos, etc.
	\end{itemize}
\end{frame}

\end{document}

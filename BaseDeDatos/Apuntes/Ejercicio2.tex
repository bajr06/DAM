\documentclass{article}
\usepackage{amsmath}
\usepackage{amssymb}

\title{04.2 Ejercicios de Álgebra Relacional}
\author{Bryan Andreu Jiménez Rojas}
\date{Febrero 2025}

\begin{document}
\maketitle

\section{Parte 1}
\text{1. Selecciona todas las personas que sean jefes e ingenieros.}
\begin{equation*}
	Ingenieros \cap Jefes
\end{equation*}

\text{2. Selecciona el nombre de todos los ingenieros que sean jefes.}
\begin{equation*}
	\Pi_{Nombre} \left(Ingenieros \cap Jefes\right)
\end{equation*}

\text{3. Selecciona el código de los ingenieros que no sean jefes}
\begin{equation*}
	\Pi_{\#E} \left(Ingenieros - Jefes\right)
\end{equation*}

\text{4. Selecciona la edad de todos los ingenieros que sean mayores de 30 que no sean jefes}
\begin{equation*}
	\Pi_{Edad} \left(\sigma_{edad>30} (Ingenieros - Jefes )\right)
\end{equation*}


\section{Parte 2}
\text{1. Selecciona todos los proyectos en los que los ingenieros se llama José.}
\begin{equation*}
	\begin{split}
		\Pi_{Proyecto} \left(\sigma \text{Nombre} = \text{José} \left(Ingenieros\right)\bowtie Departamentos \right) \\
		T1.D\# = Departamentos.D\#
	\end{split}
\end{equation*}

\text{2. Selecciona todos los proyectos que tengan mayor duración en tiempo de 30.}
\begin{equation*}
	\Pi_{Proyecto} \left(\sigma_{Tiempo > 30} \left(Proyecto\right)\right)
\end{equation*}

\text{3. Selecciona el nombre de todos los ingenieros y el código del departamento.}
\begin{equation*}
	\Pi_{Nombre},\#\left(Ingenieros\right)
\end{equation*}

\end{document}

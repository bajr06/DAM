\documentclass{beamer}

\usepackage[spanish]{babel}
\usepackage[utf8]{inputenc}
\usepackage[T1]{fontenc}
\usefonttheme{professionalfonts}
\usetheme{Madrid}
\usecolortheme{default}

\title{Virtualización con VMware y Ubuntu}
\subtitle{Parte 4}
\author{Víctor de Juan}
\date{\today}

\begin{document}

%----------------------------------
% Portada
%----------------------------------
\begin{frame}
    \titlepage
\end{frame}

%----------------------------------
% Índice de la Parte 4
%----------------------------------
\begin{frame}{Índice - Parte 4}
    \tableofcontents[
    hideallsubsections]
\end{frame}

%=====================================================================
% 18. Otros Hypervisores y Comparativa Rápida
%=====================================================================
\section{Otros Hypervisores y Comparativa Rápida}

%----------------------------------
\subsection{18.1 VirtualBox}
\begin{frame}{VirtualBox (I)}
    \begin{itemize}
        \item \textbf{Desarrollado por Oracle}, antes Innotek y Sun Microsystems.
        \item \textbf{Open Source Edition (OSE)} y versión propietaria con extensiones.
        \item Muy popular en entornos académicos y de desarrollo por su gratuidad y facilidad.
    \end{itemize}
\end{frame}

\begin{frame}{VirtualBox (II)}
    \begin{itemize}
        \item \textbf{Características}:
            \begin{itemize}
                \item Creación de múltiples snapshots (ilimitados).
                \item Red en modo NAT, Bridged, Host-Only, o redes internas.
                \item Soporte para “Guest Additions” (similar a VMware Tools), que ofrecen mejor integración y rendimiento.
            \end{itemize}
        \item \textbf{Ventajas}:
            \begin{itemize}
                \item Gratuito y de código abierto.
                \item Fácil de usar; interfaz intuitiva.
            \end{itemize}
        \item \textbf{Desventajas}:
            \begin{itemize}
                \item Históricamente, menor rendimiento 3D que VMware/Hyper-V (aunque ha mejorado).
                \item Menos opciones empresariales de soporte.
            \end{itemize}
    \end{itemize}
\end{frame}

%----------------------------------
\subsection{18.2 Hyper-V}
\begin{frame}{Hyper-V (I)}
    \begin{itemize}
        \item \textbf{Producto de Microsoft}, disponible en:
            \begin{itemize}
                \item Windows Server (modo bare metal / Tipo 1).
                \item Windows 10/11 Pro, Enterprise o Education (Modo “Tipo 2” sobre Windows).
            \end{itemize}
        \item Soporta virtualización de Windows y Linux con la funcionalidad de “Enhanced Session Mode”.
    \end{itemize}
\end{frame}

\begin{frame}{Hyper-V (II)}
    \begin{itemize}
        \item \textbf{Ventajas}:
            \begin{itemize}
                \item Integración nativa con Windows.
                \item Gestión remota y centralizada con herramientas Microsoft (SCVMM).
            \end{itemize}
        \item \textbf{Desventajas}:
            \begin{itemize}
                \item Incompatibilidad con algunos otros hypervisores (VT-x puede estar bloqueado si Hyper-V está habilitado).
                \item Menor compatibilidad con macOS o entornos que requieran GPU passthrough en modo escritorio.
            \end{itemize}
    \end{itemize}
\end{frame}

%----------------------------------
\subsection{18.3 KVM/QEMU}
\begin{frame}{KVM/QEMU (I)}
    \begin{itemize}
        \item \textbf{KVM} (Kernel-based Virtual Machine): Tecnología de virtualización nativa en el kernel Linux (Tipo 1 a nivel de núcleo).
        \item \textbf{QEMU} brinda emulación de hardware; junto con KVM acelera la virtualización en arquitecturas x86.
        \item Utilizado en muchas plataformas de virtualización y contenedorización (como Proxmox, OpenStack, etc.).
    \end{itemize}
\end{frame}

\begin{frame}{KVM/QEMU (II)}
    \begin{itemize}
        \item \textbf{Ventajas}:
            \begin{itemize}
                \item Código abierto y libre.
                \item Altísimo rendimiento, cercano a bare-metal en muchos casos.
                \item Flexibilidad: soporta múltiples arquitecturas (x86, ARM, PowerPC…).
            \end{itemize}
        \item \textbf{Desventajas}:
            \begin{itemize}
                \item Curva de aprendizaje mayor si se configura directamente con línea de comandos.
                \item Interface gráfica no tan “amigable” por defecto (aunque existen herramientas como \textbf{virt-manager}).
            \end{itemize}
    \end{itemize}
\end{frame}

%----------------------------------
\subsection{18.4 Proxmox VE (Enfoque Servidor)}
\begin{frame}{Proxmox VE (I)}
    \begin{itemize}
        \item \textbf{Proxmox Virtual Environment}: Plataforma basada en Debian que combina KVM para virtualización de máquinas y LXC para contenedores.
        \item Ofrece una interfaz web completa para administrar nodos, redes, almacenamiento, backups, etc.
        \item \textbf{Diseño clusterizado}: Fácil implementación de alta disponibilidad (HA), replicación y migración en vivo (live migration).
    \end{itemize}
\end{frame}

\begin{frame}{Proxmox VE (II)}
    \begin{itemize}
        \item \textbf{Características}:
            \begin{itemize}
                \item Soporte de \textbf{CEPH} y ZFS para almacenamiento distribuido o local.
                \item Gestión de contenedores (LXC) y máquinas virtuales KVM en la misma interfaz.
            \end{itemize}
        \item \textbf{Ventajas}:
            \begin{itemize}
                \item Solución “todo en uno” para virtualización y contenedores.
                \item Código abierto y una gran comunidad.
            \end{itemize}
        \item \textbf{Desventajas}:
            \begin{itemize}
                \item Enfoque principalmente \textbf{servidor}, no tan pensado para escritorio.
                \item Requiere un mínimo de conocimiento en KVM, redes, almacenamiento.
            \end{itemize}
    \end{itemize}
\end{frame}

%=====================================================================
% Comparativa General
%=====================================================================
\section{Comparativa General}

\scriptsize % Reduce el tamaño del texto de la tabla
\begin{frame}{Tabla Comparativa de algunos Hipervisores}
\begin{center}
\scriptsize % Reduce el tamaño del texto de la tabla para ajustarla mejor
\begin{tabular}{l c c c}
\toprule
 & \textbf{VirtualBox} & \textbf{Hyper-V} & \textbf{KVM/QEMU} \\
\midrule
\textbf{Licencia} & GPL (OSE) / Prop. & Propietario (MS) & GPL  \\
\textbf{Tipo} & 2 (Hosted) & 1/2 (según edición) & 1 (Kernel Linux) \\
\textbf{Uso principal} & Escritorio / Dev & Windows / Servidor & Linux / Servidor \\
\textbf{Soporte 3D} & Básico / Mejorando & Limitado & Indirecto (virtio) \\
\textbf{Facilidad de uso} & Alta (GUI simple) & Media (win-based) & Media/Alta (virt-manager) \\
\textbf{Rendimiento} & Bueno & Muy bueno & Excelente \\
\textbf{Coste} & Gratis (principal) & Incluido con Win & Gratis \\
\bottomrule
\end{tabular}
\end{center}
\end{frame}

\end{document}

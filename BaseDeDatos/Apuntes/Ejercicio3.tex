\documentclass{article}
\usepackage{amsmath}
\usepackage{amssymb}

\title{04.2 Ejercicios de Álgebra Relacional}
\author{Bryan Andreu Jiménez Rojas}
\date{Febrero 2025}

\begin{document}
\maketitle

\section{Parte 1}
\textnormal{1. Obtener datos de todas las clases}
\begin{equation*}
\begin{split}
        T1 \leftarrow Clases \bowtie Asistencia \\
        Clases.Código = Asistencia.Código_{Clase}
        \\\\
        T2 \leftarrow T1 \bowtie Asignatura \\
        T1.Código_{Asignatura} = Asignatura.Código_{Asignatura}
        \\\\
        T3 \leftarrow T2 \bowtie Profesor \\
        T2.Código_Profesor = Profesor.Código_Profesor \\
        \Pi_{Nombre}\left(Nombre = Nombre_{Profesor}\left(T3\right)\right)
\end{split}
\end{equation*}

\textnormal{2. Obtener datos de todas las clases ubicadas en el primer piso}
\begin{equation*}
\begin{split}
        T1 \leftarrow \Pi_{Código_{Clase}, Bloque} \left(\sigma_{Piso = 1}\left(Clases\right)\right)
        \\\\
        T2 \leftarrow \Pi_{Bloque, Código_{Profesor}, Código_{Asignatura}, Código_{Clase}}\left(T1 \bowtie Asistencia\right) \\
        T1.Código_{Clase} = Asistencia.Código_{Clase}
        \\\\
        \Pi_{Código_{Profesor}}\left(T2 \bowtie Profesor\right) \\
        T2.Código_{Profesor} = Profesor.Código_{Profesor}
\end{split}
\end{equation*}

\end{document}

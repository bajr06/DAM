\documentclass[10]{article}
\usepackage{amsmath}

\begin{document}
	\begin{tittle}
		\LARGE textft{01.02 EJERCICIOS DE IDENTIFICACIÓN DEL LENGUAJE DE MARCAS}
	\end{tittle}

\\ \text{Con la información recopilada escribe en código:}
\begin{equation}
	\\frac{1}{1+\sqrt{2}-\sqrt{3}}=\frac{1}{\sqrt{2}-\sqrt{3}}*\frac{\left(1+\sqrt{2}\right)+\sqrt{3}}{\left(1+\sqrt{2}\right)+\sqrt{3}}
\end {equation}

\\\\8. \text{El conjunto solución de la ecuación} \begin{equation} \sqrt{x+\sqrt{x+8}}=2\sqrt{x} \end{equation} \text{es:}

\begin{equation}
	\\\sum_{k=1}^{1}k\left(k!\right)=1\left(1!\right)=1,
\end{equation}

\\ \text{Suponemos que}\begin{equation} x\in\left(-\infty,1\right)=1-x, \end{equation} \text{pues} \begin{equation} 1-x>0.\end{equation} \text{Por lo tanto:}

\end{document}

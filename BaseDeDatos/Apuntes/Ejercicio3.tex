\documentclass{article}

\usepackage{amsmath}
\usepackage{amssymb}

\title{04.4 Ejercicios de Álgebra Relacional}
\author{Bryan Andreu Jiménez Rojas}
\date{Febrero 2025}

\begin{document}
\maketitle

\textnormal{1. Obtener datos de todas las clases}
\begin{equation*}
	\begin{split}
		T1 \leftarrow Clases \bowtie Asistencia \\
		Clases.Código = Asistencia.Código_{Clase}
		\\\\
		T2 \leftarrow T1 \bowtie Asignatura \\
		T1.Código_{Asignatura} = Asignatura.Código_{Asignatura}
		\\\\
		T3 \leftarrow T2 \bowtie Profesor \\
		T2.Código_Profesor = Profesor.Código_Profesor \\
		\Pi_{Nombre}\left(Nombre = Nombre_{Profesor}\left(T3\right)\right)
	\end{split}
\end{equation*}

\textnormal{2. Obtener datos de todas las clases ubicadas en el primer piso}
\begin{equation*}
	\begin{split}
		T1 \leftarrow \Pi_{Código_{Clase}, Bloque} \left(\sigma_{Piso = 1}\left(Clases\right)\right)
		\\\\
		T2 \leftarrow \Pi_{Bloque, Código_{Profesor}, Código_{Asignatura}, Código_{Clase}}\left(T1 \bowtie Asistencia\right) \\
		T1.Código_{Clase} = Asistencia.Código_{Clase}
		\\\\
		\Pi_{Código_{Profesor}}\left(T2 \bowtie Profesor\right) \\
		T2.Código_{Profesor} = Profesor.Código_{Profesor}
	\end{split}
\end{equation*}

\textnormal{3. Obtener los profesores que imparten clase en la clase C1}
\begin{equation*}
	\begin{split}
		T1 \leftarrow \Pi Código_{Profesor}\left(\sigma Código_{close = 'C1'}(Asistencia)\right) \\
		\Pi Nombre \left(T1 \bowtie Profesor\right) \\
		T1.Código_{Profesor} = Profesor.Código.Profesor
	\end{split}
\end{equation*}

\textnormal{4. Obtener los valores de piso y bloque para las clases que imparte el profesor con el código 1.}
 \begin{equation*}
	 \begin{split}
		T1 \leftarrow \Pi Código_{Clase} \left(\sigma Código_{Profesor = '1'}(Asistencia)\right) \\
		\Pi Piso, Bloque \left(T1 \bowtie Clases\right) \\
		T1.Código_{Clase} = Clases.Código_{Clases}
	\end{split}
\end{equation*}

\textnormal{5. Obtener los valores de piso y bloque para las clases que imparte la profesora Rosa.}
\begin{equation*}
	\begin{split}
		T1 \leftarrow \Pi Código_{Profesor} \left(\sigma Nombre = "Rosa" (Profesor)\right) \\
		T2 \leftarrow \Pi Código_{Clase} \left(T1 \bowtie Asistencia\right) \\
		T1.Código_{Profesor} = Asistencia.Código_{Profesor}
		\\\\
		\Pi Piso, Bloque \left(T2 \bowtie \text{Clases} \right) \\
		T2.Código_{Clase} = Clases.Código_{Clase}
	\end{split}
\end{equation*}

\textnormal{6. Obtener el código de profesor para los profesores que imparten clase en C1 pero que imparten la asignatura de física.}
\begin{equation*}
	\begin{split} % Considerar que la manera en que escriba esta ecuación se tendra que aplicar al resto de ejercicios.
		T1 \leftarrow \Pi \text{Código\_Asignatura} \left(\sigma \text{Nombre = "Física"}\left(\text{Asignatura}\right)\right) \\
		\Pi \text{Código\_Profesor}\left(\sigma \text{Código\_Clases = "C1"} \left(T1 \bowtie \text{Asistencia}\right)\right) \\
		\text{T1.Código\_Asignatura = Asistencia.Código\_Asistencia}
	\end{split}
\end{equation*}

\textnormal{7. Obtener los nombres de los profesores que imparten clase en C1 o C2}
\begin{equation*}
	\begin{flalign*}
		& T1 \leftarrow \Pi \text{Código\_Profesor} \left(\sigma \text{Código\_Clase = 'C1'} \wedge (or) \text{Código\_Clase = 'C2'} \left(\text{Asistencia}\right)\right) \\
		& \Pi \left(T1 \bowtie \text{Profesor}\right) \\
		& \text{T1.Código\_Profesor = Profesor.Código\_Profesor}
	\end{flalign*}
\end{equation*}

\textnormal{8. Obtener los nombres de los profesores que imparten clase en C2 y C3}
\begin{equation*}
	\begin{flalign*}
		& T1 \leftarrow \Pi \text{Código\_Profesor} \left(\sigma \text{Código\_Clase = 'C2'} \left(\text{Asistencia}\right)\right) \\
		& T2 \leftarrow \Pi \text{Código\_Profesor} \left(\sigma \text{Código\_Clase = 'C3'} \left(\text{Asistencia}\right)\right) \\
		& T3 \leftarrow \left( T1 \cap T2\right) \leftright 3 \\
		& \Pi \text{Nombre} \left( T3 \bowtie \text{Profesor} \right) \\
		& \text{T3.Código\_Profesor = Profesor.Código\_Profesor}
	\end{flalign*}
\end{equation*}

\textnormal{9. }
\begin{equation*}
	\begin{flaling*}
		& T1 \leftarrow \Pi_{Código\_Clase} \left(\sigma_{Bloque = '1'} \left(\text{Clase}\right)\right) \\
		& T2 \leftarrow \Pi_{Código\_Profesor} \left(T1 \bowtie \text{Asistencia}\right) \\
		& \text{T1.Código\_Clase = Asistencia.Código\_Clase} \\
		& \Pi_{Nombre} \left(T2 \bowtie \text{Profesor} \right) \\
		& \text{T2.Código\_Profesor = Profesor.Código\_Profesor}
	\end{flaling*}
\end{equation*}

\textnormal{}
\begin{equation*}
	\begin{flalign*}
		& T1 \leftarrow \Pi_{Código\_Asignatura, Código\_Clase} \left(\text{Asistencia}\right) \\
		& \Pi_{Código\_Clase, Nombre} \left(T1 \bowtie \text{Asignatura}\right) \\
		& \text{T1.Código\_Asignatura = Asignatura.Código\_Asignatura}
	\end{flalign*}
\end{equation*}

\textnormal{}
\begin{equation*}
	\begin{flalign*}
		& & T1 \leftarrow \Pi_{Código\_Clase} \left(\sigma_{Bloque = '2'} \left(\text{Clase}\right)\right) \\
		& T2 \leftarrow \Pi_{Código\_Asignatura, T1.Código\_Clase} \left(T1 \bowtie \text{Asistencia}\right) \\
		& \text{T1.Código\_Clase = Asistencia.Código\_Clase}
		% ESTO QUE COMPLETAR TODO
	\end{flalign*}
\end{equation*}

\end{document}
